\documentclass[11pt]{article}
\usepackage{graphicx}
\usepackage{amssymb}
\usepackage{epstopdf}
\DeclareGraphicsRule{.tif}{png}{.png}{`convert #1 `dirname #1`/`basename #1 .tif`.png}

\textwidth = 6.5 in
\textheight = 9 in
\oddsidemargin = 0.0 in
\evensidemargin = 0.0 in
\topmargin = 0.0 in
\headheight = 0.0 in
\headsep = 0.0 in



%\parskip = 0.2in
\parindent = 0.0in

\begin{document}
\begin{center}
{\large {\bf Budget Justification for Stanford University}}
\end{center}

 \vspace{-1.0em}
\subsection*{1. Personnel}


\begin{itemize}

\item Two Principal Investigators will be responsible for separate components of the project at Stanford, and will share responsibility for supervising experimental and corpus-based components of the project as well as supervision of graduate and undergraduate research personnel.
\begin{itemize}
\item PI Cleo Condoravdi is Professor (Research) of Linguistics at Stanford University.
She is committed for the following time: one and one-half summer salary months in each of the three years. 
She will have primary responsibility for the veridicity component of the project and coordinating with consultants Dr.\ Zaenen and Dr.\ Kartunnen. 

\item PI Daniel Lassiter is Assistant Professor of Linguistics at Stanford University.
He is committed for the following time: one and one-half summer salary months in each of the three years. 
He is responsible for coordinating work on scalar adjectives with Princeton PI Christiane Fellbaum.
\end{itemize}

\item Funding is requested for two undergraduate students working 10 hours per week during the academic year (30 weeks total). 
They will be responsible for implementing and conducting Mechanical Turk-based experiments, data analysis, and collecting and analyzing preliminary corpus data.

\item Two graduate students from the Computational Linguistics program are funded for full-time research for three months each summer.  
They will be responsible for using machine learning methods to analyze corpus data and results from human subjects experiments, as well as comparison with existing resources such as WordNet. 
They will also be responsible for interfacing with groups participating in the RTE task during the evaluation phase of the project.

\item Consultants Dr.\ Kartunnen and Dr.\ Zaenen have made seminal contributions to the study of inferential properties of linguistic expressions both from a theoretical and computational point of view. They were also crucially involved in the pilot experimental studies on clause-selecting adjectives reported in the proposal. Funding is requested for 20 hours of consulting per year for each one of them for three years.

\end{itemize}

\subsection*{2.  Experimental costs}
\$10,000 is requested annually for participant costs, including costs for experiments relating both to the scalar adjectives and veridicality components of the project. In line with the iterative-refinement methodology articulated in the proposal, we expect to run many versions of the experiments with the goal of developing new standards for crowdsourcing linguistic annotation.
 
%\vspace{-.05em}
\subsection*{3.  Travel}

\$10,000 is budgeted for travel for the two PIs and research personnel for the following activities:
%\vspace{-1.0em}
\begin{enumerate}
 
\item Travel to PI Meetings each year. We have budgeted \$800 per year for each PI's travel, accommodations, and meals. 
 
\item Travel for PIs to attend domestic and international conferences, and for research personnel and consultants to attend domestic conferences, in order to report on findings from the proposed work. 

\end{enumerate}

\subsection*{4.   Fringe}

The DHHS approved fringe benefit rates for Stanford University for fiscal year 2014 (July 1, 2014 through June 30, 2015) and beyond is 30.0\% for the PI (Full-time faculty) and 7.7\% for the undergraduate students. The rate for students is effective only in the 3 summer months of June, July and August. Therefore it is prorated by 25\%.    

\subsection*{5.  Overhead}

Indirect Costs � Modified total direct costs based on DHHS negotiated rates of June 26, 2012 as follows:
\\
July 1 2014 � June 30 2015 � 60.5\%\\
July 1 2015 � June 30 2016 � 60.5\%\\
July 1 2016 � June 30, 2017� 60.5\%
 



 \end{document}

